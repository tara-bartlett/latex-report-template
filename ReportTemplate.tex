\documentclass{article}

%preamble
\usepackage{pdfpages}
\usepackage{lipsum}
\usepackage[margin=1in,includefoot]{geometry}
\usepackage[hidelinks]{hyperref}
\usepackage{url}
%\usepackage{natbib}
\usepackage{graphicx}
\usepackage{float}
\usepackage[none]{hyphenat}
\usepackage{xfrac}
\usepackage{pgfplots}
\newlength\figureheight
\newlength\figurewidth	
\usepackage{gensymb}
\usepackage{pdfpages}
\usepackage{overcite}
\usepackage{tabularx}
\usepackage{amsmath}
% gets rid of spacing between items in a list
\usepackage{enumitem}
\setlist{noitemsep}
\usepackage{subcaption}
\usepackage{pdflscape}
% Get matlab code in
\usepackage[framed,numbered,autolinebreaks,useliterate]{mcode}

% Gets references in toc, can't be used with natbib
\usepackage[nottoc]{tocbibind}

% Algorithms
\usepackage[]{algorithm2e}
\usepackage{mdframed}

% Title spacing
\usepackage{titlesec}
\titlespacing\section{0pt}{12pt plus 4pt minus 2pt}{0pt plus 2pt minus 2pt}
\titlespacing\subsection{0pt}{12pt plus 4pt minus 2pt}{0pt plus 2pt minus 2pt}
\titlespacing\subsubsection{0pt}{12pt plus 4pt minus 2pt}{0pt plus 2pt minus 2pt}

% Table of contents spacing
\usepackage{tocloft}
\setlength\cftparskip{-2pt}

\begin{document}

\begin{titlepage}
	\includepdf[pages={1}]{Coversheets/coverSheet.pdf}
\end{titlepage}
\cleardoublepage	
\begin{titlepage}
	\begin{center}
		
		\vspace{2cm}
		\line(1,0){400}\\
		[1cm]
		\huge{\bfseries Title for Assignment}\\
		[2mm]
		\line(1,0){400}\\
		[1.5cm]
		\textsc{\LARGE SUBJ1234 Assignment X}\\
	\vspace{11.75cm}
		\textsc{\large Tara Bartlett 450198331}
	\end{center}
\end{titlepage}

\pagenumbering{roman}

\section*{Abstract}

\newpage
\tableofcontents
\cleardoublepage


\pagenumbering{arabic}
\setcounter{page}1

\newpage
% main body stuff

\section{Introduction}

\subsection{Aim}

\newpage
\section{Literature Review}
Before starting the project, it was important to comprehensively understand what issues are being faced by systems controlled in similar ways to $AIRUS$, and what solutions are already available.

\subsection{Inserting a figure} Figure \ref{example}.

\begin{figure}[H]
	\centering
	\includegraphics[width = 0.8\textwidth]{Figures/Example}
	\caption{Example Figure}
	\label{example}
\end{figure}

\subsection{Referencing \cite{litRev2}}
\subsection{Table}
Table \ref{Table} is a table. Use latex table generator

\begin{table}[H]
	\centering
	\raggedright
	\begin{tabular}{ | p{0.17\linewidth} |
			p{0.275\linewidth} |
			p{0.15\linewidth} |
			p{0.35\linewidth} | } 
		\hline
		\textbf{Variable Name} & \textbf{Definition} & \textbf{Initialization}   & \textbf{Application}    \\ \hline
		\textit{dtTelem}                & Time between telemetry being sent and received                     & User input to parent file & Compensated for in the prediction, used to calculate expected current position                                                                                              \\ \hline
		\textit{dtControlReceived}      & Time between control being sent and received                       & User input to parent file & Used in the prediction to find position of green dot, represents the lag in the system                                                                                      \\ \hline
		\textit{dtControlActed}         & Time taken for commanded velocity to be implemented, once received & User input to parent file & The reciprocal of this is used as the proportional gain constant in the PD controller                                                                                   \\ \hline
		\textit{dtEuler}                & Shorter time step used for Euler integration in predictor          & Defined in predictor      & Used to iterate through the displacement and velocity prediction calculations in each loop of the system                                                                \\ \hline
		\textit{dt\_i}                  & Time taken for each iteration                                      & Calculated in parent file & Represents the time taken for each iteration in the simulation, used in many functions in the system to find the number of iterations that corresponds to a given delay \\ \hline
		\textit{timeTaken}              & Total time taken for the red dot to reach the end of the plot      & Calculated in parent file & The total time taken for the pilot to move the red dot from the initial position to the final position                                                                  \\ \hline
	\end{tabular}
	\caption{Definition of Time Parameters}
	\label{Table}
\end{table}



\subsection{subfigures}

		\begin{figure}[H]
			\begin{subfigure}[b]{0.5\textwidth}
				\centering
				\includegraphics[width=\textwidth]{Figures/Example}
			\end{subfigure}
			\hfill
			\begin{subfigure}[b]{0.5\textwidth}
				\centering
				\includegraphics[width=\textwidth]{Figures/Example}
			\end{subfigure}
			\vfill
			\begin{subfigure}[b]{0.5\textwidth}
				\centering
				\includegraphics[width=\textwidth]{Figures/Example}
			\end{subfigure}
			\hfill
			\begin{subfigure}[b]{0.5\textwidth}
				\centering
				\includegraphics[width=\textwidth]{Figures/Example}
			\end{subfigure}
			\caption{Example subfigures}
			\label{subfig}
		\end{figure}

\newpage

\section{Conclusion}


\newpage
\pagenumbering{roman}
\setcounter{page}1
\bibliographystyle{aiaa}
\bibliography{Bibliography}

\newpage
\section{Appendix}
\subsection{\textsc{Matlab} Code}

\noindent\textbf{Figure Formatting}

\lstinputlisting{MATLAB/formatFigure.m}

\end{document}